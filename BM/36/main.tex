%\iffalse
\let\negmedspace\undefined
\let\negthickspace\undefined
\documentclass[journal,12pt,twocolumn]{IEEEtran}
\usepackage{cite}
\usepackage{amsmath,amssymb,amsfonts,amsthm}
\usepackage{algorithmic}
\usepackage{graphicx}
\usepackage{textcomp}
\usepackage{xcolor}
\usepackage{txfonts}
\usepackage{listings}
\usepackage{enumitem}
\usepackage{mathtools}
\usepackage{gensymb}
\usepackage{comment}
\usepackage[breaklinks=true]{hyperref}
\usepackage{tkz-euclide} 
\usepackage{listings}
\usepackage{gvv}                                        
\def\inputGnumericTable{}                                 
\usepackage[latin1]{inputenc}                                
\usepackage{color}                                            
\usepackage{array}                                            
\usepackage{longtable}                                       
\usepackage{calc}                                             
\usepackage{multirow}                                         
\usepackage{hhline}                                           
\usepackage{ifthen}                                           
\usepackage{lscape}
\usepackage{circuitikz}
\newtheorem{theorem}{Theorem}[section]
\newtheorem{problem}{Problem}
\newtheorem{proposition}{Proposition}[section]
\newtheorem{lemma}{Lemma}[section]
\newtheorem{corollary}[theorem]{Corollary}
\newtheorem{example}{Example}[section]
\newtheorem{definition}[problem]{Definition}
\newcommand{\BEQA}{\begin{eqnarray}}
\newcommand{\EEQA}{\end{eqnarray}}
\newcommand{\define}{\stackrel{\triangle}{=}}
\theoremstyle{remark}
\newtheorem{rem}{Remark}
\begin{document}

\bibliographystyle{IEEEtran}
\vspace{3cm}

\title{DISCRETE}
\author{EE23BTECH11006 - Ameen Aazam$^{*}$% <-this % stops a space
}
\maketitle
\newpage
\bigskip

\renewcommand{\thefigure}{\theenumi}
\renewcommand{\thetable}{\theenumi}

\vspace{3cm}
\textbf{Question :}
The time-dependent growth of a bacterial population is governed by the equation
\begin{align}
    \frac{dx}{dt}=x\brak{1-\frac{x}{200}}
\end{align}
where $x$ is the population size at time $t$. The initial population size is $x_0=100$
at $x=0$. As $t \rightarrow \infty$, the population size of bacteria asymptotically approaches
\begin{enumerate}[label=(\alph*)]
    \item $150$
    \item $200$
    \item $300$
    \item $500$
\end{enumerate}
\hfill{(GATE BM 2023)}

\solution
%\fi
The growth equation is given by,
\begin{align}
    &\frac{dx(t)}{dt}=x(t)\brak{1-\frac{x(t)}{200}} \\
    \implies &\int_{0}^{t}d[x(t)]=\int_{0}^{t}x(t)dt-\frac{1}{200}\int_{0}^{t}x^2(t)dt \\
\end{align}
So the corresponding growth curve is,
\begin{figure}[h]
    \centering
    \includegraphics[width=01.1\columnwidth]{figs/CT.png}
    \caption{$x(t) vs t$}
\end{figure}
\newline
Now approximating by trapezoidal rule of integration between $t_{n-1}$ to $t_n$ with the stepsize being $h$ we have,
\begin{align}
    &x(t_n)-x(t_{n-1})=\frac{h}{2}\sbrak{x(t_n)+x(t_{n-1})-\frac{1}{200}(x^2(t_n)+x^2(t_{n-1})}
\end{align}
Next, replacing $t_{n}=hn$ we get the difference equation,
\begin{align}
    &x(hn)=\frac{-\brak{1-\frac{h}{2}}+\sqrt{\brak{1-\frac{h}{2}}^2+\frac{h}{100}p_1}}{h/200}
\end{align}
Where,
\begin{align}
    p_1=\sbrak{\brak{1-\frac{h}{2}}x(h(n-1))-\frac{h}{400}x^2(h(n-1))}
\end{align}
And relabeling $hn\longleftrightarrow n$ we will be having the discrete time equation as,
\begin{align}
    x(n)=\frac{-\brak{1-\frac{h}{2}}+\sqrt{\brak{1-\frac{h}{2}}^2+\frac{h}{100}p_2}}{h/200} \label{eq:A}
\end{align}
Where,
\begin{align}
    p_2=\sbrak{\brak{1-\frac{h}{2}}x(n-1)-\frac{h}{400}x^2(n-1)}
\end{align}
Now, plotting the difference equation,
\begin{figure}[h]
    \centering
    \includegraphics[width=01.2\columnwidth]{figs/DT.png}
    \caption{Approximation in Discrete-Time}
\end{figure}
\newline
As we can see the discrete time plot is following the actual curve, which means \eqref{eq:A} is indeed a good approximation of the original continuous-time equation.
\newline
And the population size approaches to 200 as $t \rightarrow \infty$.
\end{document}
